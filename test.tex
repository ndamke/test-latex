\documentclass[12pt, a4paper]{article}
\usepackage[german,ngerman]{babel}
\usepackage[ansinew]{inputenc}
\usepackage[T1]{fontenc}
\usepackage{scrpage2}
\usepackage{listings}
\usepackage{color}
\usepackage{graphicx}
\usepackage{fancyhdr} %Paket laden
\usepackage{layout}    %  um die Seitenränder als Bild auszugeben
\usepackage{geometry}

\geometry{
   left=3cm,
   textwidth=15cm,
   marginpar=3cm}

\layout
\pagestyle{fancy}
\fancyhf{} %alle Kopf- und Fußzeilenfelder bereinigen
\fancyhead[C]{Titel} %Kopfzeile links
\fancyfoot[C]{\thepage} %Seitennummer


\definecolor{grund}{gray}{.9}           % grauer Box-Hintergrund

\begin{document}
\author{Norbert Damke \verb+damke@gmx.de}
\title{Mein erstes \LaTeX-Dokument}
\date{18. Mai 1999}
\maketitle
\section*{Erstes Kapitel}
\subsection*{Unterkapitel 1}

Hier beginnt nun unser erstes wunderbares LaTeX-Dokument,
das die grundlegenden Eingaben zeigen soll \dots
\lstinputlisting[language=PHP,backgroundcolor=\color{grund}]{test.php}
\fbox{\includegraphics[frame=single,width=1\textwidth]{testalt.pdf}} %70% der Textbreiteter
Dies steht unter dem Bild
\begin{align*}
  \sqrt{x^4} &= x^2\\
  \lim_{n\to\infty} \frac 1{n^2} &= 0 \\
  \int_{-1}^2 x\, \mathrm{d}x &= \left[ \frac12 x^2 \right]_1^2
\end{align*}

\end{document}
